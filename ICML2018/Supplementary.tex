%%%%%%%% ICML 2018 EXAMPLE LATEX SUBMISSION FILE %%%%%%%%%%%%%%%%%

\documentclass{article}

% Recommended, but optional, packages for figures and better typesetting:
\usepackage{microtype}
\usepackage{graphicx}
\usepackage{subfigure}
\usepackage{booktabs} % for professional tables

% hyperref makes hyperlinks in the resulting PDF.
% If your build breaks (sometimes temporarily if a hyperlink spans a page)
% please comment out the following usepackage line and replace
% \usepackage{icml2018} with \usepackage[nohyperref]{icml2018} above.
\usepackage{hyperref}

% Attempt to make hyperref and algorithmic work together better:
\newcommand{\theHalgorithm}{\arabic{algorithm}}

% Use the following line for the initial blind version submitted for review:
\usepackage{icml2018}
\usepackage{natbib}
\usepackage{amsmath}
\usepackage{amsthm}
\usepackage{amsfonts}
\usepackage{bbm}
\usepackage{bm}
\usepackage{amssymb}
\newtheorem{theorem}{Theorem}
\newtheorem{proposition}[theorem]{Proposition}
\newtheorem{definition}{Definition}
\newtheorem{lemma}[theorem]{Lemma}
%\makeatletter
%\renewenvironment{proof}[1][\proofname]{\par
%  \vspace{-\topsep}% remove the space after the theorem
%  \pushQED{\qed}%
%  \normalfont
%  \topsep0pt \partopsep0pt % no space before
%  \trivlist
%  \item[\hskip\labelsep
%        \itshape
%    #1\@addpunct{.}]\ignorespaces
%}{%
%  \popQED\endtrivlist\@endpefalse
%  \addvspace{0pt plus 0pt} % some space after
%}
%\makeatother

% If accepted, instead use the following line for the camera-ready submission:
%\usepackage[accepted]{icml2018}

% The \icmltitle you define below is probably too long as a header.
% Therefore, a short form for the running title is supplied here:
\icmltitlerunning{Supplementary File}

\begin{document}

\twocolumn[
\icmltitle{Supplementary File for\\Incentivizing High Quality Crowdsourcing Information using Bayesian Inference and Reinforcement Learning} 

% It is OKAY to include author information, even for blind
% submissions: the style file will automatically remove it for you
% unless you've provided the [accepted] option to the icml2018
% package.

% List of affiliations: The first argument should be a (short)
% identifier you will use later to specify author affiliations
% Academic affiliations should list Department, University, City, Region, Country
% Industry affiliations should list Company, City, Region, Country

% You can specify symbols, otherwise they are numbered in order.
% Ideally, you should not use this facility. Affiliations will be numbered
% in order of appearance and this is the preferred way.
\icmlsetsymbol{equal}{*}

\begin{icmlauthorlist}
\icmlauthor{Zehong Hu}{NTU}
\icmlauthor{Yang Liu}{Harvard}
\icmlauthor{Yitao Liang}{UCLA}
\icmlauthor{Jie Zhang}{NTU}
\end{icmlauthorlist}

%\icmlaffiliation{NTU}{Rolls-Royce Cooperate Lab@NTU, School of Computer Science and Engineering, Nanyang Technological University, Singapore}
%\icmlaffiliation{goo}{Googol ShallowMind, New London, Michigan, USA}
%\icmlaffiliation{ed}{School of Computation, University of Edenborrow, Edenborrow, United Kingdom}

%\icmlcorrespondingauthor{Cieua Vvvvv}{c.vvvvv@googol.com}
%\icmlcorrespondingauthor{Eee Pppp}{ep@eden.co.uk}

% You may provide any keywords that you
% find helpful for describing your paper; these are used to populate
% the "keywords" metadata in the PDF but will not be shown in the document
%\icmlkeywords{Machine Learning, ICML}

\vskip 0.3in
]

% this must go after the closing bracket ] following \twocolumn[ ...

% This command actually creates the footnote in the first column
% listing the affiliations and the copyright notice.
% The command takes one argument, which is text to display at the start of the footnote.
% The \icmlEqualContribution command is standard text for equal contribution.
% Remove it (just {}) if you do not need this facility.

%\printAffiliationsAndNotice{}  % leave blank if no need to mention equal contribution
%\printAffiliationsAndNotice{} % otherwise use the standard text.

\begin{lemma}
\label{MoGene}
If $x\sim \mathrm{Bin}(n,p)$, $\mathbb{E}t^x= \left(1-p+tp\right)^{n}$ holds for any $t>0$, where $\mathrm{Bin}(\cdot)$ is the binomial distribution.
\begin{proof}
\begin{equation}
t^x = e^{x\log t}=m_x(\log t)= \left(1-p+pe^{\log t}\right)^{n}
\end{equation}
where $m_x(\cdot)$ denotes the moment generating function.
\end{proof}
\end{lemma}

\begin{lemma}
\label{SolveF}
For given $n,m\geq 0$, if $0\leq p\leq 1$, we can have
\begin{equation*}
\begin{split}
&{\sum}_{x=0}^{n}{\sum}_{w=0}^{m} C_{n}^{x}C_{m}^{w}p^{x+w}(1-p)^{y+z}\times\\
&\qquad\qquad\qquad B(x+z+1+t,y+w+1)=\\
&\int_{0}^{1}[(2p-1)x+1-p]^{n}[(1-2p)x+p]^{m}x^{t}\mathrm{d}x
\end{split}
\end{equation*}
\begin{proof}
By the definition of the beta function~\cite{olver2010nist},
\begin{equation}
B(x, y) = \int_{0}^{+\infty} u^{x-1}(1+u)^{-(x+y)}\mathrm{d}u
\end{equation}
we can have
\begin{align}
&\sum_{x,w} C_{n}^{x}C_{m}^{w}p^{x+w}(1-p)^{y+z}B(x+z+1+t,y+w+1)\nonumber\\
&= \int_{0}^{+\infty} \mathbb{E}u^{x}\cdot\mathbb{E}u^z \cdot u^t\cdot (1+u)^{-(n+m+2+t)}\mathrm{d}u
\end{align}
where we regard $x\sim \mathrm{Bin}(n,p)$ and $z\sim \mathrm{Bin}(m,1-p)$.
Thus, according to Lemma~\ref{MoGene}, we can obtain
\begin{equation}
\begin{split}
&\int_{0}^{+\infty} \mathbb{E}u^{x}\cdot\mathbb{E}u^z \cdot u^t\cdot (1+u)^{-(n+m+3)}\mathrm{d}u\\
&=\int_{0}^{+\infty} \frac{[1-p+up]^n\cdot [p+(1-p)u]^m\cdot u^t}{(1+u)^{n+m+2+t}}\mathrm{d}u.
\end{split}
\end{equation}
For the integral operation, substituting $u$ with $v-1$ at first and then $v$ with $(1-x)^{-1}$, we can conclude Lemma~\ref{SolveF}.
\end{proof}
\end{lemma}

\begin{lemma}
\label{Sum1}
$\sum_{n=0}^{N}C_N^{n}\cdot x^{n}=(1+x)^N$.
\end{lemma}
\begin{lemma}
\label{Sum2}
$\sum_{n=0}^{N} C_N^{n}\cdot n\cdot x^{n}=N\cdot x\cdot (1+x)^{N-1}$.
\end{lemma}
\begin{lemma}
\label{Sum3}
$\sum_{n=0}^{N} C_N^{n}\cdot n\cdot x^{N-n}=N\cdot (1+x)^{N-1}$.
\end{lemma}
\begin{lemma}
\label{Sum5}
$\sum_{n=0}^{N} C_N^{n}\cdot n^2\cdot x^{n}=Nx(1+Nx)(1+x)^{N-2}$.
\end{lemma}
\begin{lemma}
\label{Sum6}
$\sum_{n=0}^{N} C_N^{n}\cdot n^2\cdot x^{N-n}=N(x+N)(1+x)^{N-2}$.
\end{lemma}
\begin{lemma}
\label{Sum4}
If $0<x<1$, we can have
\begin{equation*}
\begin{split}
\sum_{n=0}^{\lfloor N/2\rfloor} C_N^{n}\cdot x^{n} &\geq \left(1-e^{-cN}\right) \cdot (1+x)^{N}\\
\sum_{n=\lfloor N/2 \rfloor +1}^{N} C_N^{n}\cdot x^{N-n}&\geq \left(1-e^{-cN}\right) \cdot (1+x)^{N}.
\end{split}
\end{equation*}
where $c=0.5(1-x)^2(1+x)^{-2}$.
\begin{proof}
To prove the lemmas above, we firstly define
\begin{equation}
F_t(x)=\sum_{n=0}^{N} C_N^{n} n^t x^{n}
\end{equation}
Then, Lemma~\ref{Sum1} can be obtained by expanding $(1+x)^N$.
Lemma~\ref{Sum2} can be proved as follows
\begin{equation}
\begin{split}
F_1(x)&=\sum_{n=0}^{N} C_N^{n} (n+1) x^{n}-(1+x)^N\\
\sum_{n=0}^{N} C_N^{n} (n+1) x^{n}&=\frac{\mathrm{d}}{\mathrm{d}x}\left[xF_0(x)\right]\\
&=Nx(1+x)^{N-1}+(1+x)^N.
\end{split}
\end{equation}
Lemma~\ref{Sum3} can be obtained as follows
\begin{equation}
\begin{split}
\sum_{n=0}^{N} C_N^{n} n x^{N-n}&=x^N\sum_{n=0}^{N} C_N^{n} n \left(\frac{1}{x}\right)^n\\
&=x^N\cdot N\cdot \frac{1}{x}\cdot \left(1+\frac{1}{x}\right)^{N-1}.
\end{split}
\end{equation}
For Lemma~\ref{Sum5}, we can have
\begin{align}
F_2(x)&=\sum_{n=0}^{N} C_N^{n} (n+2)(n+1) x^{n}-3F_{1}(x)-2F_{0}(x)\nonumber \\
&=\left[x^2F_0(x)\right]' -3F_{1}(x)-2F_{0}(x)
\end{align}
Thus, we can have
\begin{equation}
F_2(x)=Nx(1+Nx)(1+x)^{N-2}
\end{equation}
which concludes Lemma~\ref{Sum5}.
Then, Lemma~\ref{Sum6} can be obtained by considering Equation~\ref{Eqv}.
\begin{equation}
\label{Eqv}
\begin{split}
\sum_{n=0}^{N} C_N^{n} n^2 x^{N-n}=x^N\sum_{n=0}^{N} C_N^{n} n^2 \left(\frac{1}{x}\right)^n.
\end{split}
\end{equation}
For Lemma~\ref{Sum4}, we can have
\begin{equation}
\sum_{n=0}^{\lfloor N/2\rfloor} C_N^{n} x^{n}=(1+x)^{N}\sum_{n=0}^{\lfloor N/2\rfloor} C_N^{n} p^n (1-p)^{N-n}
\end{equation}
where $p=x(1+x)^{-1}$. Let $X\sim \mathrm{Bin}(N, p)$, we can have
\begin{equation}
\sum_{n=0}^{\lfloor N/2\rfloor} C_N^{n} p^n (1-p)^{N-n}\geq 1-P\left(X\geq N/2\right).
\end{equation}
Since $x<1$, $p<0.5$ and $Np<N/2$. Considering Hoeffding's inequality, we can get
\begin{equation}
P\left(X\geq N/2\right)\leq \exp \left[-\frac{N(1-x)^2}{2(1+x)^2}\right]
\end{equation}
which concludes the first inequality in Lemma~\ref{Sum4}. Similarly, for the second inequality, we can have
\begin{equation}
\sum_{n=K}^{N} C_N^{n}x^{N-n}=(1+x)^{N}\sum_{n=K}^{N} C_N^{n} (1-p)^n p^{N-n}
\end{equation}
where $K=\lfloor N/2 \rfloor +1$. Suppose $Y\sim \mathrm{Bin}(N, 1-p)$, we can have
\begin{equation}
\sum_{n=K}^{N} C_N^{n} (1-p)^n p^{N-n}\geq 1-P\left(Y\leq N/2\right).
\end{equation}
Considering Hoeffding's inequality, we can also get
\begin{equation}
P\left(Y\leq N/2\right)\leq \exp \left[-\frac{N(1-x)^2}{2(1+x)^2}\right]
\end{equation}
which concludes the second inequality in Lemma~\ref{Sum4}.
\end{proof}
\end{lemma}

\begin{lemma}
\label{Inequality1}
For any $x,y\geq 0$, we can have
$$(1+x)^{y}\leq e^{xy}.$$
\begin{proof}
Firstly, we can know $(1+x)^{y}=e^{y\log(1+x)}$. Let $f(x)=x-\log(x)$. Then, we can have $f(0)=0$ and $f'(x)\geq 0$. Thus, $x\geq\log (1+x)$ and we can conclude Lemma~\ref{Inequality1} by taking this inequality into the equality.
\end{proof}
\end{lemma}

\begin{lemma}
\label{Concave1}
$$g(x)=\frac{e^x}{e^x+1}$$
is a concave function when $x\in [0,+\infty)$.
\begin{proof}
$g'(x)= (2+t(x))^{-1}$, where $t(x)=e^x+e^{-x}$. $t'(x)=e^x-e^{-x}\geq 0$ when $x\in [0,+\infty)$. Thus, $g'(x)$ is monotonically decreasing when $x\in [0,+\infty)$, which concludes Lemma~\ref{Concave1}.
\end{proof}
\end{lemma}

\begin{lemma}
\label{Concave2}
For $x\in(-\infty, +\infty)$, 
$$h(x)=\frac{1}{e^{|x|}+1}$$
satisfies
$$h(x)< e^x \;\;\mathrm{and}\;\; h(x)< e^{-x}.$$
\begin{proof}
When $x\geq 0$, we can have
\begin{equation}
h(x)<\frac{1}{e^x}=e^{-x}\leq e^{x}.
\end{equation}
When $x\leq 0$, we can have
\begin{equation}
h(x)=\frac{e^x}{e^x+1}<e^{x}\leq e^{-x}.
\end{equation}
\end{proof}
\end{lemma}

\begin{lemma}
\label{Concave3}
If $\lambda=p/(1-p)$ and $0.5<p<1$, then
$${\sum}_{n=\lfloor N/2 \rfloor}^{N}C_N^n\lambda^{m-n}p^n(1-p)^m\leq [4p(1-p)]^{N/2}$$
$${\sum}_{n=0}^{\lfloor N/2 \rfloor}C_N^n\lambda^{n-m}p^n(1-p)^m\leq [4p(1-p)]^{N/2}$$
where $m=N-n$.
\begin{proof}
For the first inequality, we can have
\begin{align}
&\sum_{n=\lfloor N/2 \rfloor}^{N}C_N^n\lambda^{m-n}p^n(1-p)^m \\&=\sum_{n=\lfloor N/2 \rfloor}^{N}C_N^np^m(1-p)^n\leq \sum_{m=0}^{\lfloor N/2 \rfloor}C_N^m p^m(1-p)^n\nonumber
\end{align}
According to the inequality in \cite{Arratia1989}, we can have
\begin{equation}
\label{xxx}
 \sum_{m=0}^{\lfloor N/2 \rfloor}C_N^m p^m(1-p)^n \leq \exp (-ND)
\end{equation}
where $D=-0.5\log(2p)-0.5\log(2(p-1))$, which concludes the first inequality in Lemma~\ref{Concave3}.

For the second inequality, we can have
\begin{equation}
\begin{split}
&\sum_{n=0}^{\lfloor N/2 \rfloor}C_N^n\lambda^{n-m}p^n(1-p)^m \\
&= \frac{1}{[p(1-p)]^N}\sum_{n=0}^{\lfloor N/2 \rfloor}C_N^n[p^3]^n[(1-p)^3]^m
 \\&=\frac{[p^3+(1-p)^3]^N}{[p(1-p)]^N}\sum_{n=0}^{\lfloor N/2 \rfloor}C_N^n x^n(1-x)^m
\end{split}
\end{equation}
where $x=p^3/[p^3+(1-p)^3]$. By using Equation~\ref{xxx}, we can have
\begin{equation}
\begin{split}
&\sum_{n=0}^{\lfloor N/2 \rfloor}C_N^n\lambda^{n-m}p^n(1-p)^m\\
&\leq \frac{[p^3+(1-p)^3]^N}{[p(1-p)]^N} [x(1-x)]^{N/2} \\
&=[4p(1-p)]^{N/2}
\end{split}
\end{equation}
which concludes the second inequality of Lemma~\ref{Concave3}.
\end{proof}
\end{lemma}
\bibliographystyle{named}
\bibliography{ref}

\end{document}


% This document was modified from the file originally made available by
% Pat Langley and Andrea Danyluk for ICML-2K. This version was created
% by Iain Murray in 2018. It was modified from a version from Dan Roy in
% 2017, which was based on a version from Lise Getoor and Tobias
% Scheffer, which was slightly modified from the 2010 version by
% Thorsten Joachims & Johannes Fuernkranz, slightly modified from the
% 2009 version by Kiri Wagstaff and Sam Roweis's 2008 version, which is
% slightly modified from Prasad Tadepalli's 2007 version which is a
% lightly changed version of the previous year's version by Andrew
% Moore, which was in turn edited from those of Kristian Kersting and
% Codrina Lauth. Alex Smola contributed to the algorithmic style files.
