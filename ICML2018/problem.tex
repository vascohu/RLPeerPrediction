\section{Problem Formulation}
\label{PF}
Suppose one data requester assigns $M$ tasks with binary answer space $\left\{1,2\right\}$ to $N \geq 3$ candidate workers at each time step $t$.
We denote the tasks and workers by $\mathcal{T}^{t}=\{1,2,\ldots,M\}$ and $\mathcal{C}=\{1,2,\ldots,N\}$, respectively.
Meanwhile, we use $L^{t}_{i}(j)$ to denote the label generated by worker $i\in \mathcal{C}$ for task $j\in\mathcal{T}^{t}$.
If $L^{t}_{i}(j)=0$, we mean that task $j$ is not assigned to worker $i$ at step $t$.

The generated label $L^{t}_{i}(j)$ depends both on the ground-truth label $L^{t}(j)$ and worker $i$'s effort level $e^{t}_i$ and reporting strategy $r^{t}_i$.
Any worker $i$ can potentially have two effort levels, High ($e^{t}_i=1$) and Low ($e^{t}_i=0$).
Also, he/she can decide either to truthfully report his observation $r^{t}_i = 1$ or to revert the answer $r^{t}_i = 0$.
Workers may act differently for different tasks. 
We thus define $e^{t}_i\in[0,1]$ and $r^{t}_i\in[0,1]$ as worker $i$'s probability of exerting high efforts and being truthful, respectively.
In this case, worker $i$'s probability of being correct (PoBC) can be computed as
\begin{equation}
\begin{split}
&p^{t}_i=r^{t}_i e^{t}_i p_{i, H}+r^{t}_i (1-e^{t}_i) p_{i, L}+\\
&(1-r^{t}_i) e^{t}_i (1-p_{i, H})+(1-r^{t}_i) (1-e^{t}_i) (1-p_{i, L})
\end{split}
\end{equation}
where $p_{i, H}$ and $p_{i, L}$ denote worker $i$'s probability of observing the correct label when exerting high and low efforts, respectively.
Following \cite{dasgupta2013crowdsourced,liu2017sequential}, we assume that the tasks are homogeneous and the workers share the same set of $p_{i, H}, p_{i, L}$, denoting by $p_H, p_L$, and $p_{H}>p_{L}= 0.5$.
Here, $p^t_i=0.5$ means that worker $i$ randomly selects a label to report.

The data requester needs to pay each worker some money as the incentive for providing labels.
We denote the payment for worker $i$ at step $t$ as $P^{t}_{i}$.
At the beginning of each time step, the data requester promises the workers a certain rule of payment determination which acts the contract between two sides and cannot be changed until the next time step.
The workers are self-interested and may change their reporting strategies ($e^t_i$ and $r^t_i$) according to the payment rule.
Workers' different reporting strategies will lead to the different values of workers' PoBCs and finally different levels of label quality.
After collecting the labels from the workers, the data requester will infer the true labels by using a certain inference algorithm, and \cite{zheng2017truth} provide a good survey of existing inference algorithms.
Denote the the inferred true label of task $j$ by $\tilde{L}^{t}(j)$.
Then, the label accuracy $A^t$ and the utility $u^t$ of the data requester satisfy
\begin{equation}
\label{utility}
\begin{split}
A^t&=\frac{1}{M}{\sum}_{j=1}^{M}1\left[\tilde{L}^{t}(j)=L^{t}(j)\right]\\
u^t &= F(A^t) - \eta {\sum}_{i=1}^{N}P^t_i
\end{split}
\end{equation}
where $F(\cdot)$ is a non-decreasing monotonic function mapping accuracy to utility and $\eta$ is a
tunable parameter balancing label quality and costs. Intuitively, the $F(\cdot)$ function needs to be non-deceasing as higher accuracy is preferred.

The number of tasks in crowdsourcing is often very large, and the interaction between tasks and workers may last for hundreds of time steps.
Thus, we introduce the cumulative utility $U(t)$ of the data requester from the current step $t$ as
\begin{equation}
U(t)={\sum}_{k=t}^{\infty}\rho^{k-t}u^t
\end{equation}
where $0\leq \rho< 1$ is the discount factor which determines the importance of future utilities.
The objective of our study is to maximize $U(t)$ by optimally designing the payment rule and the ex-post adjustment algorithm of the payment rule, which we call as the incentive mechanism.